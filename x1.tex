\documentclass[conference]{IEEEtran}
\IEEEoverridecommandlockouts
\usepackage{cite}
\usepackage{amsmath,amssymb}
\usepackage{graphicx}
\usepackage{textcomp}
\usepackage{xcolor}
\usepackage{hyperref}

\begin{document}

\title{Related Work: DRL-Based Adaptive SFC Deployment in Cloud-Native Data Centers}

\author{
\IEEEauthorblockN{Your Name}
\IEEEauthorblockA{
\textit{Department of Computer Science} \\
\textit{Your University} \\
City, Country \\
email@domain.com}
}

\maketitle

\section{Related Work}

The evolution of data centers has laid the foundation for intelligent and agile network function deployment. In early work, Liao et al. [1] proposed a software-defined inter-data center approach to reduce east-west traffic and improve efficiency through centralized orchestration. Moghaddam et al. [2] expanded on this by envisioning fully cloud-native and service-based architectures that decompose legacy monoliths into microservices, highlighting the importance of observability and modular scalability. These advancements in data center intelligence inspire our vision of using learned representations and adaptive agents to dynamically manage service function chains (SFCs).

As infrastructures advanced, attention turned to how services are chained and deployed. Traditional SFC deployment methods were often based on heuristic or rule-based placement. Bari et al. [3] addressed VNF orchestration using resource-aware algorithms, while Mohammadkhan et al. [4] focused on simplifying service chaining through segment routing-based abstractions. Despite their effectiveness in static environments, these approaches fall short under fluctuating workloads or in environments with stringent latency and bandwidth constraints. This motivated a shift towards models that can learn from experience and adapt in real time.

Deep Reinforcement Learning (DRL) emerged as a powerful candidate for dynamic SFC deployment. Li et al. [5] proposed a delay-aware DRL model using variable action sets to better handle SFC scheduling. Qi et al. [6] extended this line of work by integrating Graph Neural Networks (GNNs) with DRL to model SFCs as graphs and learn representations that generalize across unseen service requests. However, while these models demonstrate strong adaptability, they often optimize limited objectives such as delay or placement success, neglecting broader goals like energy efficiency, SLA compliance, and predictive scaling. Our work introduces a multi-layered reward mechanism to address these gaps, combining resource-layer feedback, SLA constraints, and global operator-level incentives.

Automation technologies such as Kubernetes further enable scalable and reactive orchestration. Arouk and Nikaein [7] presented an approach to automate 5G deployment via Kubernetes and OpenShift Operators, supporting dynamic RAN reconfiguration. Wiranata et al. [8] pushed this further by implementing Mosaic5G with Kubernetes for full network slicing automation and real-time monitoring using ElasticSearch and Kibana. These systems provide the control plane capabilities necessary for intelligent agents to make decisions and trigger actions. Our architecture builds on this by integrating SARIMA-based forecasting into the monitoring loop, enabling vertical auto-scaling based on predicted traffic patterns and observed label metrics such as energy, utilization, and upgrade state.

Together, these works reflect a clear evolution: from rigid, deterministic systems to programmable infrastructures, from handcrafted placement rules to learning-based optimization, and from reactive scaling to predictive adaptation. Our proposed framework sits at the convergence of these efforts—offering a GNN-DRL-based system that dynamically places microservices, evaluates performance using layered rewards, and proactively adjusts capacity using time-series forecasts.

\begin{thebibliography}{99}

\bibitem{datacenter1}
D.~J.~Liao, H.~Xu, and Y.~Jiang, ``Software-defined Inter-data Center Traffic Engineering,'' \textit{Journal of Communications and Networks}, vol.~21, no.~4, pp.~396--405, 2019.

\bibitem{datacenter2}
S.~M.~Moghaddam and A.~Leon-Garcia, ``Software-defined Cloud-native Architectures: Vision and Research Challenges,'' in \textit{Proc. CNSM}, 2020.

\bibitem{sfc1}
M.~Bari et al., ``On Orchestrating Virtual Network Functions,'' in \textit{Proc. CNSM}, 2015.

\bibitem{sfc2}
A.~Mohammadkhan et al., ``Service Function Chaining Simplified,'' in \textit{Proc. ACM SOSR}, 2017.

\bibitem{drl1}
J.~Li et al., ``Delay-aware VNF Scheduling: A Reinforcement Learning Approach with Variable Action Set,'' \textit{IEEE Transactions on Cognitive Communications and Networking}, vol.~6, no.~4, pp.~1203--1217, 2020.

\bibitem{drl2}
S.~Qi et al., ``Energy-Efficient VNF Deployment for Graph-Structured SFC Based on Graph Neural Network and Constrained Deep Reinforcement Learning,'' in \textit{Proc. APNOMS}, 2021.

\bibitem{kube1}
O.~Arouk and N.~Nikaein, ``5G Cloud-Native: Network Management and Automation,'' in \textit{Proc. IEEE/IFIP NOMS}, 2020.

\bibitem{kube2}
F.~A.~Wiranata et al., ``Automation of Virtualized 5G Infrastructure Using Mosaic 5G Operator over Kubernetes Supporting Network Slicing,'' in \textit{Proc. IEEE ICIT}, 2020.

\end{thebibliography}

\end{document}
