\documentclass[conference]{IEEEtran}
\IEEEoverridecommandlockouts
\usepackage{cite}
\usepackage{amsmath,amssymb}
\usepackage{graphicx}
\usepackage{textcomp}
\usepackage{xcolor}
\usepackage{hyperref}

\begin{document}

\title{Related Work: DRL-Based Adaptive SFC Deployment in Cloud-Native Data Centers}

\author{
\IEEEauthorblockN{Your Name}
\IEEEauthorblockA{
\textit{Department of Computer Science} \\
\textit{Your University} \\
City, Country \\
email@domain.com}
}

\maketitle

\section{Related Work}

The transformation of data centers into intelligent, traffic-aware environments has laid the foundation for dynamic service deployment. Paul et al. [1] introduced a regional data center optimization framework that combines K-means clustering and RNN-based traffic forecasting to support 5G scalability and reliability. This dynamic modeling of traffic patterns provides early evidence for integrating predictive intelligence into deployment strategies. Expanding the perspective, Tzanakaki et al. [2] proposed a converged infrastructure integrating optical, wireless, and data center networks. Their hierarchical orchestration model emphasizes the need for cross-domain coordination in next-generation service architectures. These two works shape the premise that data center design is no longer static but deeply tied to traffic behavior and heterogeneous infrastructure collaboration.

With data center evolution underway, efficient Service Function Chain (SFC) placement emerges as the next challenge. Harutyunyan et al. [3] tackled this with a latency-aware SFC placement strategy that formulates the problem as an ILP and proposes heuristics to address real-time deployment. In a complementary study, the same authors extended their work by incorporating user mobility into the decision process, allowing the placement mechanism to dynamically adapt as users move through the network [4]. Both approaches highlight critical limitations of static placement—namely, their inability to generalize across scenarios with real-time performance demands or user-centric constraints.

To bridge these limitations, Deep Reinforcement Learning (DRL) has been introduced as a viable alternative. Qi et al. [5] proposed an energy-efficient deployment strategy for graph-structured SFCs using GNN-based embeddings and constrained DRL, offering generalization across topologies and workloads. However, while this method achieves structural learning, it does not directly address latency-bounded services or cross-domain constraints. Addressing this, Dalgkitsis et al. [6] introduced SCHE2MA, a scalable DRL framework capable of minimizing both energy and latency in multi-domain ultra-reliable low-latency (URLLC) contexts. Together, these DRL-based works support the case for learning-driven, resource- and SLA-aware SFC orchestration in complex environments.

To operationalize intelligent deployment, automation frameworks like Kubernetes have become central. Arouk and Nikaein [7] demonstrated how 5G networks can be dynamically reconfigured using OpenShift Operators within Kubernetes, enabling lifecycle management of monolithic and disaggregated RANs. Building on this, Wiranata et al. [8] integrated Mosaic5G and FlexRAN within a Kubernetes stack to achieve end-to-end slicing and visualization through ElasticSearch and Kibana. These frameworks provide the substrate for real-time orchestration, where intelligent agents can observe, decide, and act based on system feedback. Our proposed work builds atop this operational layer by incorporating SARIMA-based forecasting into the monitoring loop, enabling vertical auto-scaling that anticipates traffic demands and proactively adjusts resource allocation.

Together, these studies mark a transition—from static heuristics and isolated management tools to predictive, learning-driven, and orchestrated service deployments. Our framework synthesizes these advances into a unified architecture, coupling GNN-based DRL with monitoring, forecasting, and orchestration to deliver scalable and adaptive SFC deployment in cloud-native data centers.

\begin{thebibliography}{99}

\bibitem{dc1}
U.~Paul, S.~Dey, and M.~Chatterjee, ``Traffic-Profile and Machine Learning Based Regional Data Center Design and Operation for 5G Network,'' \textit{Journal of Communications and Networks}, vol.~21, no.~4, pp.~406--416, 2019.

\bibitem{dc2}
A.~Tzanakaki et al., ``Converged Optical, Wireless and Data Centre Network Infrastructures for 5G Services,'' \textit{Journal of Lightwave Technology}, vol.~37, no.~12, pp.~3140--3148, 2019.

\bibitem{sfc1}
D.~Harutyunyan et al., ``Latency-Aware Service Function Chain Placement in 5G Mobile Networks,'' in \textit{Proc. IEEE ICC}, 2019.

\bibitem{sfc2}
D.~Harutyunyan et al., ``Latency and Mobility–Aware Service Function Chain Placement in 5G Networks,'' in \textit{Proc. IEEE GLOBECOM}, 2020.

\bibitem{drl1}
S.~Qi et al., ``Energy-Efficient VNF Deployment for Graph-Structured SFC Based on Graph Neural Network and Constrained Deep Reinforcement Learning,'' in \textit{Proc. APNOMS}, 2021.

\bibitem{drl2}
A.~Dalgkitsis et al., ``SCHE2MA: A Scalable and Energy-Aware SFC Orchestration Approach for Multi-Domain B5G URLLC Services,'' in \textit{Proc. IEEE NOMS}, 2020.

\bibitem{kube1}
O.~Arouk and N.~Nikaein, ``5G Cloud-Native: Network Management and Automation,'' in \textit{Proc. IEEE/IFIP NOMS}, 2020.

\bibitem{kube2}
F.~A.~Wiranata et al., ``Automation of Virtualized 5G Infrastructure Using Mosaic 5G Operator over Kubernetes Supporting Network Slicing,'' in \textit{Proc. IEEE ICIT}, 2020.

\end{thebibliography}

\end{document}
